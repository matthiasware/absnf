%------------------------------------------------------------------------
\begin{frame}
	\frametitle{Content}
	\framesubtitle{ABS-Normal Form}
	\begin{columns}[T] % align columns
		\begin{column}{.48\textwidth}
			
			\begin{center}
				{\Huge Solve}
			\end{center}
			
		\end{column}%
		\hfill%
		\begin{column}{.48\textwidth}
			\color{blue}\rule{\linewidth}{4pt}
			
			\setbeamertemplate{enumerate items}[default]
			\begin{enumerate}
				\item Einführung
				\item Aufgaben
				\item Evaluate
				\item Gradient
				\item Blocksize und Gridsize
				\item \textbf{Solve}
				\item Final Thoughts
			\end{enumerate}
		\end{column}%
	\end{columns}
\end{frame}
%------------------------------------------------------------------------
\begin{frame}
	\frametitle{Solve}
	\framesubtitle{Aufgaben}
	Gegben: 
	\begin{flalign*}
		a,b,Z,L,J,Y,\Delta y
	\end{flalign*}
	Gesucht:
	\begin{flalign*}
		\Delta x, \Delta z
	\end{flalign*}
	\begin{flalign*}
	\begin{pmatrix}
	\Delta z \\
	\Delta y
	\end{pmatrix}
	= 
	\begin{pmatrix}
	a \\
	b
	\end{pmatrix}
	+
	\begin{pmatrix}
	Z & L \\
	J & Y 
	\end{pmatrix}
	\times
	\begin{pmatrix}
	\Delta x \\
	|\Delta z |
	\end{pmatrix}
	\end{flalign*}
\end{frame}
%------------------------------------------------------------------------
\begin{frame}
	\frametitle{Solve}
	\framesubtitle{Lösungsansätze}
	Wollen Lösen:
	\begin{flalign*}
		\Delta z &= a + Z \Delta x + L |\Delta z| \\
		\Delta y &= b + J \Delta x + Y |\Delta z| \\
	\end{flalign*}
	Ausgangslage:
	\begin{flalign*}
		\Delta y = 0
	\end{flalign*}
	Andernfalls
	\begin{flalign*}
		b' = b - \Delta y
	\end{flalign*}
	\pause
	Umstellen nach $\Delta x$:
\begin{flalign*}
\Delta y &= b + J \Delta x + Y |\Delta z| \\
0 &= b + J \Delta x + Y |\Delta z| \\
- b - Y |\Delta z| &= J \Delta x \\
b + Y |\Delta z| &= J \Delta x (-1) \\
J^{-1}(b + Y |\Delta z|) &= - \Delta x \\
\Delta x = - J^{-1}(b + Y |\Delta z|)
\end{flalign*}
\end{frame}
%------------------------------------------------------------------------
\begin{frame}
	\frametitle{Solve}
	\framesubtitle{Lösungsansätze}
	Wollen Lösen:
	\begin{flalign*}
	\Delta z &= a + Z \Delta x + L |\Delta z| \\
	\Delta y &= b + J \Delta x + Y |\Delta z| \\
	\end{flalign*}
	Haben:
	\begin{flalign*}
	\Delta x = - J^{-1}(b + Y |\Delta z|)
	\end{flalign*}
	\pause
	Einsetzen in
\begin{flalign*}
\Delta z &= a + Z \Delta x + L |\Delta z| \\
&= a + Z \Big( - J^{-1}(b + Y |\Delta z|) \Big) +  L |\Delta z| \\
&= a + Z \Big( -J^{-1}b - J^{-1}Y|\Delta z| \Big) +  L |\Delta z| \\
&= a - ZJ^{-1}b - Z J^{-1}Y|\Delta z| +  L |\Delta z| \\
&= a - ZJ^{-1}b - (Z J^{-1}Y - L)|\Delta z|
\end{flalign*}
\end{frame}
%------------------------------------------------------------------------
\begin{frame}
	\frametitle{Solve}
	\framesubtitle{Lösungsansätze}
	Wollen Lösen:
	\begin{flalign*}
	\Delta z &= a + Z \Delta x + L |\Delta z| \\
	\Delta y &= b + J \Delta x + Y |\Delta z| \\
	\end{flalign*}
	Haben:
	\begin{flalign*}
	\Delta x &= - J^{-1}(b + Y |\Delta z|) \\
	\Delta z &= a - ZJ^{-1}b - (Z J^{-1}Y - L)|\Delta z|
	\end{flalign*}
	Modularisieren:
	\begin{flalign*}
	\Delta z &= a - ZJ^{-1}b - (Z J^{-1}Y - L)|\Delta z| \\
	&= c + S|\Delta z|
	\end{flalign*}
	\begin{flalign*}
		c		 &= a - ZJ^{-1}b \\
		S		 &= L - Z J^{-1}Y
	\end{flalign*}
\end{frame}
%------------------------------------------------------------------------
\begin{frame}
	\frametitle{Solve}
	\framesubtitle{Fixpunktiteration}
	Problem:
	\begin{flalign*}
	\Delta z &= c + S|\Delta z|
	\end{flalign*}
	\pause
	Lösung mithilfe Fixpunktiteration: \\
	\begin{enumerate}
		\item Generalized (Pseudo) Newton
		\item Block-Seidel Algorithmus
		\item Modulus Iteration Algorithmus
	\end{enumerate}
	Konvergieren unter gewissen Konvergenzkrieterien (linear / endlich).
\end{frame}
%------------------------------------------------------------------------
\begin{frame}[fragile]
	\frametitle{Solve}
	\framesubtitle{Moulus Iteration Algorithmus}
	Modulus Iteration Algorithmus:
	\begin{lstlisting}[mathescape=true]
		$\Delta z$ = Init()
		$c = a - ZJ^{-1}b $
		$S = L - Z J^{-1}Y$
		while not converged:
				 $\Delta_z$ = $c + S|\Delta z|$
		$\Delta x = - J^{-1}(b + Y |\Delta z|)$
	\end{lstlisting}
	Notes:
	\pause
	\begin{itemize}
		\item Problem ist die Berechnung von $c$ und $S$
		\item $J$ nicht singulär.
	\end{itemize}

\end{frame}
%------------------------------------------------------------------------
\begin{frame}[fragile]
	\frametitle{Solve}
	\framesubtitle{Die Berechnung von $c$ und $S$}
	
	Für
	\begin{flalign*}
		S = L - Z J^{-1}Y
	\end{flalign*}
	QR - Zerlegung:
	\pause
	\begin{flalign*}
		J^{-1} Y = X \\
		Y = J X \\
		Y = QR X \\
		QY = R X
	\end{flalign*}
	Berechne:
	\begin{flalign*}
		S = L - Z X
	\end{flalign*}
	
\end{frame}
%------------------------------------------------------------------------
\begin{frame}
	\frametitle{Solve}
	\framesubtitle{Komplexität}
	Komplexität $(m=n=s)$ bei $k$ Iterationen\\
	\begin{center}
		\begin{tabular}{ l | l}
			Funktion & Komplexität Seriell \\
			\hline
			$cusolverDnDgeqrf$ & $s^3$ (QR) \\
			$cusolverDnDormqr$ & $s^2$ (QRxB) \\
			$cublasDgemm()$& $s^3$ \\
			$cublasDgemv()$ & $s^2 * k$  \\
			$cudaMemcpy()$ & $s * k$  \\
		\end{tabular} 	\\~\\
	\end{center}
\end{frame}