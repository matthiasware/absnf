%------------------------------------------------------------------------
\begin{frame}
	\frametitle{Content}
	\framesubtitle{ABS-Normal Form}
	\begin{columns}[T] % align columns
		\begin{column}{.48\textwidth}
			
			\begin{center}
				{\Huge Final Thoughts}
			\end{center}
			
		\end{column}%
		\hfill%
		\begin{column}{.48\textwidth}
			\color{blue}\rule{\linewidth}{4pt}
			
			\setbeamertemplate{enumerate items}[default]
			\begin{enumerate}
				\item Einführung
				\item Aufgaben
				\item Evaluate
				\item Gradient
				\item Blocksize und Gridsize
				\item Solve
				\item \textbf{Final Thoughts}
			\end{enumerate}
		\end{column}%
	\end{columns}
\end{frame}
%------------------------------------------------------------------------
\begin{frame}[fragile]
	\frametitle{Final Thoughts}
	\framesubtitle{Was ist da?}
Was ist da nach der ersten Implementation?
\pause
\begin{lstlisting}
-------------------------------------------------------------------------------
Language                     files          blank        comment           code
-------------------------------------------------------------------------------
Python                          25            285            486           7451
C/C++ Header                     5             82            209           1256
C++                              7             41             19            334
-------------------------------------------------------------------------------
SUM:                            37            408            714           9041
-------------------------------------------------------------------------------
\end{lstlisting}
\pause
Dabei:
\begin{itemize}
	\item Working prototype in CUDA C++ und Python
	\item Unittests
	\item Coole Plot Generatoren
\end{itemize}
\end{frame}
%------------------------------------------------------------------------
\begin{frame}
	\frametitle{Final Thoughts}
	\framesubtitle{Was fehlt?}
	Was fehlt:
	\begin{itemize}
		\item <2-> Refactoring
		\item <3-> Funktionierende Implementierung des Gen. Newton solvers
		\item <4->Useability
		\item <5-> Anwendung
		\item <6->Numerische Checks der Ergebnisse bei größeren Daten
		\item <7-> Speichermanager
		\item <8-> Spezielle Wahl für Gridsize und Blocksize
		\item <9-> Multidevice Support
		\item <10-> Sparsity
		\item <11-> Math Tuning
	\end{itemize}
\end{frame}
%------------------------------------------------------------------------
\begin{frame}
	\frametitle{Quellen}
	\framesubtitle{ect.}
	\begin{itemize}
		\item Archiv Torsten Bosse
		\item Linear Algebra and its Applications - Griewank
		\item Cuda DOC
		\item $https://castingoutnines.wordpress.com/2010/01/12/piecewise-linear-calculus-part-2-getting-to-smoothness/$
	\end{itemize}
	
\end{frame}
%------------------------------------------------------------------------
\begin{frame}
	\frametitle{Quellen}
	\framesubtitle{Fixpunktiteration}
	\begin{center}
		\animategraphics[height=6cm, width=10cm,controls]{1}{img/n/n}{0}{4}
	\end{center}
\end{frame}