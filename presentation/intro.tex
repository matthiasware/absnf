\begin{frame}
\frametitle{Inhalt}
\framesubtitle{...}
	\begin{enumerate}
		\item ABS - Normal Form
		\item Aufgabengeschreibung
		\item Implementierung
		\begin{enumerate}
			\item Verwendete Bibliotheken
			\item Evaluate
			\item Gradient
			\item Solve
		\end{enumerate}
		\item Anwendung
		\item Numerische Tests und Vergleiche
		\item Cool Snippets
		\item Imprevements
		\item Problems Lesson Learned
	\end{enumerate}
\end{frame}
%------------------------------------------------------------------------
\begin{frame}
	\frametitle{ABS-NF}
	\framesubtitle{Einführung}
	\setbeamertemplate{enumerate items}[default]
	\begin{mydef*}
		Smooth funcion \\
		A smooth function $f(x)$ is continuous and has a continuous derivative
	\end{mydef*}
	$f(x) = x^2$ $f'(x) = 2x$

\end{frame}
%------------------------------------------------------------------------
\begin{frame}
	\frametitle{ABS-NF}
	\framesubtitle{Einführung}
	\setbeamertemplate{enumerate items}[default]
	\begin{mydef*}
		Picewise smooth function \\
		A picewise smooth function $f(x)$ is made up of finitly many smooth functions, separated by
		jump discontiuities.
	\end{mydef*}
	\begin{flalign*}
		f(x) = \begin{cases}
			1 & x < -1 \\
			x^2 + 1 & -1 <= x < 2 \\
			x & 2 <= x
		\end{cases}
	\end{flalign*}
	Wir betrachten ausschließlich continuous picewise smooth functions.
\end{frame}
%------------------------------------------------------------------------
\begin{frame}
\frametitle{ABS-NF}
\framesubtitle{Einführung}
	Picewise smooth  funcitons können durch picewise linear functions (PL) approximiert werden.
	\begin{center}
		BILD
	\end{center}
	Betrachten ausschließlich continuous PL
\end{frame}
%------------------------------------------------------------------------
\begin{frame}
	\frametitle{ABS-NF}
	\framesubtitle{Einführung}
	\setbeamertemplate{enumerate items}[default]
	ABS-Normal Form:
	\begin{center}
		Repräsentierung für Picewise linear functions (PL) 
	\end{center}
	\begin{flalign*}
	\begin{pmatrix}
	\Delta z \\
	\Delta y
	\end{pmatrix}
	= 
	\begin{pmatrix}
	a \\
	b
	\end{pmatrix}
	+
	\begin{pmatrix}
	Z & L \\
	J & Y 
	\end{pmatrix}
	\times
	\begin{pmatrix}
	\Delta x \\
	|\Delta z |
	\end{pmatrix}
	\end{flalign*}
\end{frame}
%------------------------------------------------------------------------
\begin{frame}
	\frametitle{ABS-NF}
	\framesubtitle{Einführung}
	\setbeamertemplate{enumerate items}[default]
	Vorgehen:
	\begin{itemize}
		\item Any picewise linear scalar function has a so called min-max repräsentation
		\item All min max expressions can be expressed in terms of abs functions
		\item Picewise linearization wird erreicht durch algorithmic differentiation
	\end{itemize}
\end{frame}
%------------------------------------------------------------------------
\begin{frame}
	\frametitle{ABS-NF}
	\framesubtitle{Beispiel}
	\setbeamertemplate{enumerate items}[default]
	\begin{flalign*}
		F(x_1,x_2) &= (x_2^2 - x_1^+)^+ \\
			 (i)^+ &= \max(0, i)
	\end{flalign*}
	\begin{center}
		BILD
	\end{center}
	
\end{frame}
%------------------------------------------------------------------------
\begin{frame}
	\frametitle{ABS-NF}
	\framesubtitle{Einführung}
	\setbeamertemplate{enumerate items}[default]
	\begin{center}
		ADOL C, Beispiel
	\end{center}
\end{frame}
%------------------------------------------------------------------------
\begin{frame}
	\frametitle{ABS-NF}
	\framesubtitle{Aufgabenbeschreibung}
	\setbeamertemplate{enumerate items}[default]
	\begin{itemize}
		\item Evaluate abs normal form
		\item Calculate Gradient
		\item Solve abs-normal form system
	\end{itemize}
\end{frame}
%------------------------------------------------------------------------