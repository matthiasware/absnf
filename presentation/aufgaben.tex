\begin{frame}
	\frametitle{ABS-NF}
	\framesubtitle{Aufgabenbeschreibung}
	\setbeamertemplate{enumerate items}[default]
	\begin{flalign*}
	\begin{pmatrix}
	\Delta z \\
	\Delta y
	\end{pmatrix}
	= 
	\begin{pmatrix}
	a \\
	b
	\end{pmatrix}
	+
	\begin{pmatrix}
	Z & L \\
	J & Y 
	\end{pmatrix}
	\times
	\begin{pmatrix}
	\Delta x \\
	|\Delta z |
	\end{pmatrix}
	\end{flalign*}
	\begin{itemize}
		\item Evaluate abs normal form:
			\begin{itemize}
				\item Geg: $a,b,Z,L,J,Y,\Delta x$
				\item Ges: $\Delta z, \Delta y$
			\end{itemize}
		\item Calculate Gradient
			\begin{itemize}
				\item Geg: $a,b,Z,L,J,Y, \Delta Z$
				\item Ges: Gradient $\gamma, \Gamma$
			\end{itemize}
		\item Solve abs-normal form system
		\begin{itemize}
			\item Geg: $a,b,Z,L,J,Y,\Delta y$
			\item Ges: $\Delta x, \Delta Z$
		\end{itemize}
	\end{itemize}
\end{frame}
%------------------------------------------------------------------------
\begin{frame}
	\frametitle{Annahmen und Voraussetzung und Info zur Implementierung}
	\framesubtitle{Implementierung}
	\setbeamertemplate{enumerate items}[default]
	Programmiersprachen
	\begin{itemize}
		\item Python 3.5: Prototyping und Serial Performance benchmarks
		\item Cuda C++: Implementierung der paralleln ABSNF Aufgaben
	\end{itemize}
	
	Annahmen:
	\begin{enumerate}
		\item Global memory der GPU ist groß genug um alle benötigten Datenstrukturen zeitgleich zu halten
		\item Daten werden vektorisiert übergeben
		\item Benutzen Lineare Algebra so weit wie möglich
		\item Sofern möglich mappe alle Problem auf existierende Librariers
	\end{enumerate}
	
\end{frame}
%------------------------------------------------------------------------
\begin{frame}
	\frametitle{Annahmen und Voraussetzung und Info zur Implementierung}
	\framesubtitle{Implementierung}
	\setbeamertemplate{enumerate items}[default]
	Programmiersprachen
	\begin{itemize}
		\item Python 3.5: Prototyping und Serial Performance benchmarks
		\item Cuda C++: Implementierung der paralleln ABSNF Aufgaben
	\end{itemize}
	
	Annahmen:
	\begin{enumerate}
		\item Global memory der GPU ist groß genug um alle benötigten Datenstrukturen zeitgleich zu halten
		\item Daten werden vektorisiert übergeben
		\item Benutzen Lineare Algebra so weit wie möglich
		\item Sofern möglich mappe alle Problem auf existierende Librariers
	\end{enumerate}
	
\end{frame}
%------------------------------------------------------------------------

\begin{frame}
\frametitle{Projekstruktur}
\framesubtitle{Aufteilung und verwaltung von Code}
\setbeamertemplate{enumerate items}[default]

\begin{itemize}
	\item utils.hpp test\_utils.hpp
	\item absnf.h test\_absnf.cu
	\item cuutils.h test\_cuutils.cu
	\item tabsnf.h test\_tabsnf.h
	\item make
	\item absnf.py
\end{itemize}

\end{frame}
%------------------------------------------------------------------------
\begin{frame}
	\frametitle{Evaluate}
	\framesubtitle{Implementierung}
	\setbeamertemplate{enumerate items}[default]
	Benutzte Libraries:
	\begin{itemize}
		\item cuBLAS (cuda Basic Linear Algebra Subprograms)
		\begin{itemize}
			\item Matrix Vector operations
			\item Matrix Matrix operations
		\end{itemize}
		\item cuSOLVER
		\begin{itemize}
			\item Matrix factorization
			\item Triangular solve
		\end{itemize}
		\item C++ STL
	\end{itemize}
	
\end{frame}
%------------------------------------------------------------------------