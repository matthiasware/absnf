%------------------------------------------------------------------------
\begin{frame}
	\frametitle{Aufgabe 2) Gradient}
	\framesubtitle{Aufgabenbeschreibung}
	\setbeamertemplate{enumerate items}[default]
	Gegeben:
	\begin{flalign*}
		a,b,Z,L,J,Y,m,n,s, \Delta Z
	\end{flalign*}
	Gesucht:
	\begin{flalign*}
		\gamma, \Gamma
	\end{flalign*}
	Wobei:
	\begin{flalign*}
		\gamma &= b + Y \Sigma(I-L\Sigma)^{-1} a \\
		\Gamma &= J + Y \Sigma(I-L\Sigma)^{-1} Z
	\end{flalign*}
	\begin{flalign*}
		\Sigma = Diag(Sign(\Delta z))
	\end{flalign*}
	\begin{flalign*}
	\begin{pmatrix}
	\Delta z \\
	\Delta y
	\end{pmatrix}
	= 
	\begin{pmatrix}
	a \\
	b
	\end{pmatrix}
	+
	\begin{pmatrix}
	Z & L \\
	J & Y 
	\end{pmatrix}
	\times
	\begin{pmatrix}
	\Delta x \\
	|\Delta z |
	\end{pmatrix}
	\end{flalign*}
	
\end{frame}
%------------------------------------------------------------------------
\begin{frame}
	\frametitle{Aufgabe 2) Gradient}
	\framesubtitle{Interssanter Teil}
	\setbeamertemplate{enumerate items}[default]
	Brauchen:
	\begin{flalign*}
		\Sigma(I-L\Sigma)^{-1}
	\end{flalign*}
	\begin{flalign*}
		\Sigma = Diag(Sign(\Delta z))
	\end{flalign*}
	Fallstricken:
	\begin{itemize}
		\item Sparse Matrix $\Sigma$
		\item Inverse $(I-L\Sigma)^{-1}$
	\end{itemize}
	
\end{frame}
%------------------------------------------------------------------------
\begin{frame}
	\frametitle{Aufgabe 2) Gradient}
	\framesubtitle{Beispiel}
	\setbeamertemplate{enumerate items}[default]
	
	Sei:
	\begin{flalign*}
		\Delta z = [-3, 0, 4,  -1]
	\end{flalign*}
	Dann gilt für $I - L\Sigma$:
	\begin{flalign*} 
	I -
	\left(\begin{array}{cccc}
	0 		& 0 	  & 0  & 0 \\
	L_{2,1} & 0 	  & 0  & 0 \\
	L_{3,1} & L_{3,2} & 0  & 0\\
	L_{4,1} & L_{4,2} & L_{4,3} & 0 \\
	\end{array}\right) \times
	\left(\begin{array}{cccc}
	-1 & 0 & 0 & 0 \\
	0 & 0 & 0 & 0 \\
	0 & 0 & 1 & 0 \\
	0 & 0 & 0 & -1 \\
	\end{array}\right)
	= 
	\left(\begin{array}{cccc}
	1 & 0 & 0 & 0 \\
	-L_{2,1} & 1 & 0 & 0 \\
	-L_{3,1} & 0 & 1 & 0 \\
	-L_{4,1} & 0 & 0 & 1 \\
	\end{array}\right)
	\end{flalign*}
	das entspricht folgenden Operationen:
	\begin{itemize}
		\item Hinzufügen einer Hauptdiagonalen
		\item Skalieren der Spalten von $L$ mit den Vorzeichen von $\Delta z$
	\end{itemize}
	Besser dieses als Operation zu implementieren.
	Das Auflösen der unteren Dreiecksmatrix $(I-L\Sigma)^{-1}$ übernimmt CUBLAS.
\end{frame}
%------------------------------------------------------------------------
\begin{frame}
	\frametitle{Aufgabe 2) Gradient}
	\framesubtitle{Komplexität}
	BLOCKSIZE, GRIDSIZE OPTIMIZATION \\
	ROW FORMAT, COL FORMAT \\
	INTERFACE ??? \\
\end{frame}
%------------------------------------------------------------------------
\begin{frame}
	\frametitle{Aufgabenbeschreibung}
	\framesubtitle{Aufgabenbeschreibung - Solve}
	\setbeamertemplate{enumerate items}[default]
	
	\begin{flalign*}
	\left(\begin{array}{cccc}
	\tikzmarkin[ver=style cyan]{col 1}\x  & \x  & \tikzmarkin[ver=style green]{col 2} \x & \x \\
	0   & \x  & \x & \x \\
	0   & 0   & \x & \x \\
	0   & 0   & 0  & \x \\
	a \tikzmarkend{col 1}  &  b  &  c  \tikzmarkend{col 2} &  d \\
	\end{array}\right)
	\end{flalign*}
	\begin{flalign*}
	\begin{pmatrix}
	\Delta z \\
	\Delta y
	\end{pmatrix}
	= 
	\begin{pmatrix}
	a \\
	b
	\end{pmatrix}
	+
	\begin{pmatrix}
	Z & L \\
	J & Y 
	\end{pmatrix}
	\times
	\begin{pmatrix}
	\Delta x \\
	|\Delta z |
	\end{pmatrix}
	\end{flalign*}
	\begin{flalign*}
	\left(\begin{array}{c}
	\Delta z_1 \\
	\vdots \\
	\Delta z_s \\
	\Delta y_1 \\
	\vdots \\
	\Delta y_m
	\end{array}\right)  =
	\left(\begin{array}{c}
	a_1 \\
	\vdots \\
	a_s \\
	b_1 \\
	\vdots \\
	b_m \\
	\end{array}\right) +
	\left(\begin{array}{cccccc}
	Z_{1,1} & \dots & Z_{1,n} & 0 & \dots  & 0 \\
	\vdots & \ddots & \vdots  & L_x & \ddots & 0 \\
	Z_{s,1} & \dots & Z_{s,n} & L_x & L_x & 0 \\
	J_{1,1} & \dots & J_{1,n} & Y_{1,1} & \dots & Y_{1,s} \\
	\vdots  & \ddots & \vdots & \vdots & \ddots & \vdots \\
	J_{m,1} & \dots  & J_{m,n} & Y_{m,1} & \dots & Y_{m,s} \\
	\end{array}\right) \times
	\left(\begin{array}{c}
	\Delta x_1 \\
	\vdots \\
	\Delta x_n \\
	|\Delta z_1 | \\
	\vdots \\
	| \Delta z_s| \\
	\end{array}\right)
	\end{flalign*}
	
\end{frame}
%------------------------------------------------------------------------
\begin{frame}
\frametitle{Implementierung}
\framesubtitle{Client Side Storage}
	 BLOCKSIZE, GRIDSIZE OPTIMIZATION \\
	 ROW FORMAT, COL FORMAT \\
	 INTERFACE ??? \\
\end{frame}
%------------------------------------------------------------------------