\documentclass{article}

%-----------------------------------------------------------------------------
%	Packages
%-----------------------------------------------------------------------------
\usepackage[utf8]{inputenc}	% für Umlaute ect.
\usepackage{fancyhdr} % für header
\usepackage{lastpage} % für footer
\usepackage{extramarks} % für header und footer
\usepackage{amsthm} % math stuff
\usepackage{amsmath} % math stuff
\usepackage{amssymb} % math stuff
\usepackage{color}
\usepackage{listings} % code listings
\usepackage{enumitem}
\usepackage{graphicx} % für graphics
\usepackage{hyperref}
\usepackage{caption}
\usepackage{subcaption}
\usepackage{multicol}
\usepackage{pgf}
\usepackage{tikz}
\usetikzlibrary{arrows,automata} 

%-----------------------------------------------------------------------------
%	Allegmeine Dokumentsettings
%-----------------------------------------------------------------------------
% Margins
\topmargin=-0.45in
\evensidemargin=0in
\oddsidemargin=0in
\textwidth=6.5in
\textheight=9.0in
\headsep=0.25in

%Header und Footer
\pagestyle{fancy}
\chead{\moduleTitle}  
\rhead{}
\lfoot{\lastxmark}
\cfoot{} 
\rfoot{\ \thepage\ of\ \protect\pageref{LastPage}}
\renewcommand\headrulewidth{0.4pt} % Size of the header rule
\renewcommand\footrulewidth{0.4pt} % Size of the footer rule

% Zeilenabstand
\linespread{1.0} 

%----------------------------------------------------------------------------------------
%	Commands and Environments
%----------------------------------------------------------------------------------------

%---- Allows vertical lines in matrices
\makeatletter
\renewcommand*\env@matrix[1][*\c@MaxMatrixCols c]{%
	\hskip -\arraycolsep
	\let\@ifnextchar\new@ifnextchar
	\array{#1}}
\makeatother
%--------------------------

\newcommand{\moduleTitle}{Projektarbeit - Graphische Modelle}
\newcommand{\moduleSubTitle}{Theoretische Informatík I}
\newcommand{\moduleClassInstructor}{Prof. Joachim Giesen}
\newcommand{\university}{FSU Jena}
\newcommand{\moduleSemester}{SS 17}
\newcommand{\authorName}{Matthias Mitterreiter}

% Für Code listings
\lstset{ %
	language=Python,                % choose the language of the code
	basicstyle=\footnotesize,       % the size of the fonts that are used for the code
	numbers=left,                   % where to put the line-numbers
	numberstyle=\footnotesize,      % the size of the fonts that are used for the line-numbers
	stepnumber=1,                   % the step between two line-numbers. If it is 1 each line will be numbered
	numbersep=5pt,                  % how far the line-numbers are from the code
	backgroundcolor=\color{white},  % choose the background color. You must add \usepackage{color}
	showspaces=false,               % show spaces adding particular underscores
	showstringspaces=false,         % underline spaces within strings
	showtabs=false,                 % show tabs within strings adding particular underscores
	frame=single,           % adds a frame around the code
	tabsize=2,          % sets default tabsize to 2 spaces
	captionpos=b,           % sets the caption-position to bottom
	breaklines=true,        % sets automatic line breaking
	breakatwhitespace=false,    % sets if automatic breaks should only happen at whitespacee
	escapeinside={\%*}{*)}          % if you want to add a comment within your code
}

% An environment for stpes, cases ect. 
% From: http://tex.stackexchange.com/questions/32798/a-step-by-step-environment
	\newenvironment{steps}[1]{\begin{enumerate}[label=#1 \arabic*]}{\end{enumerate}}
\makeatletter%
\def\step{%
	\@ifnextchar[ \@step{\@noitemargtrue\@step[\@itemlabel]}}
\def\@step[#1]{\item[#1]\mbox{}\\\hspace*{\dimexpr-\labelwidth-\labelsep}}
\makeatother

%http://tex.stackexchange.com/questions/10669/how-to-enumerate-equations
\def\Item$#1${\item $\displaystyle#1$
   \hfill\refstepcounter{equation}(\theequation)}

%-----------------------------------------------------------------------------
%	Theoreme
%-----------------------------------------------------------------------------
\theoremstyle{definition}
\newtheorem{mydef}{Definition}
\newtheorem*{mydef*}{Definition}
\newtheorem{mybei}{Beispiel}
\newtheorem*{mybei*}{Beispiel}
\newtheorem{mysatz}{Satz}
\newtheorem*{mysatz*}{Satz}
\newtheorem{mybew}{Beweis}

\newtheorem*{mybew*}{Beweis}
\newtheorem{myfolg}{Folgerung}
\newtheorem*{myfolg*}{Folgerung}
\newtheorem{mybemerk}{Bemerkung}
\newtheorem*{mybemerk*}{Bemerkung}

% Boxed Theoreme

\newenvironment{fmylemma*}
  {\begin{mdframed}\begin{mylemma*}}
  {\end{mylemma*}\end{mdframed}}

\newenvironment{fmykorollar*}
  {\begin{mdframed}\begin{mykorollar*}}
  {\end{mykorollar*}\end{mdframed}}

\newenvironment{fmysatz*}
  {\begin{mdframed}\begin{mysatz*}}
  {\end{mysatz*}\end{mdframed}}

\newenvironment{fmylemma}
  {\begin{mdframed}\begin{mylemma}}
  {\end{mylemma}\end{mdframed}}

\newenvironment{fmykorollar}
  {\begin{mdframed}\begin{mykorollar}}
  {\end{mykorollar}\end{mdframed}}

\newenvironment{fmysatz}
  {\begin{mdframed}\begin{mysatz}}
  {\end{mysatz}\end{mdframed}}

\begin{document}
	\begin{frame}
\frametitle{Inhalt}
\framesubtitle{...}
	\begin{enumerate}
		\item ABS - Normal Form
		\item Aufgabengeschreibung
		\item Implementierung
		\begin{enumerate}
			\item Verwendete Bibliotheken
			\item Evaluate
			\item Gradient
			\item Solve
		\end{enumerate}
		\item Anwendung
		\item Numerische Tests und Vergleiche
		\item Cool Snippets
		\item Imprevements
		\item Problems Lesson Learned
	\end{enumerate}
\end{frame}
%------------------------------------------------------------------------
\begin{frame}
	\frametitle{ABS-NF}
	\framesubtitle{Einführung}
	\setbeamertemplate{enumerate items}[default]
	\begin{mydef*}
		Smooth funcion \\
		A smooth function $f(x)$ is continuous and has a continuous derivative
	\end{mydef*}
	$f(x) = x^2$ $f'(x) = 2x$

\end{frame}
%------------------------------------------------------------------------
\begin{frame}
	\frametitle{ABS-NF}
	\framesubtitle{Einführung}
	\setbeamertemplate{enumerate items}[default]
	\begin{mydef*}
		Picewise smooth function \\
		A picewise smooth function $f(x)$ is made up of finitly many smooth functions, separated by
		jump discontiuities.
	\end{mydef*}
	\begin{flalign*}
		f(x) = \begin{cases}
			1 & x < -1 \\
			x^2 + 1 & -1 <= x < 2 \\
			x & 2 <= x
		\end{cases}
	\end{flalign*}
	Wir betrachten ausschließlich continuous picewise smooth functions.
\end{frame}
%------------------------------------------------------------------------
\begin{frame}
\frametitle{ABS-NF}
\framesubtitle{Einführung}
	Picewise smooth  funcitons können durch picewise linear functions (PL) approximiert werden.
	\begin{center}
		BILD
	\end{center}
	Betrachten ausschließlich continuous PL
\end{frame}
%------------------------------------------------------------------------
\begin{frame}
	\frametitle{ABS-NF}
	\framesubtitle{Einführung}
	\setbeamertemplate{enumerate items}[default]
	ABS-Normal Form:
	\begin{center}
		Repräsentierung für Picewise linear functions (PL) 
	\end{center}
	\begin{flalign*}
	\begin{pmatrix}
	\Delta z \\
	\Delta y
	\end{pmatrix}
	= 
	\begin{pmatrix}
	a \\
	b
	\end{pmatrix}
	+
	\begin{pmatrix}
	Z & L \\
	J & Y 
	\end{pmatrix}
	\times
	\begin{pmatrix}
	\Delta x \\
	|\Delta z |
	\end{pmatrix}
	\end{flalign*}
\end{frame}
%------------------------------------------------------------------------
\begin{frame}
	\frametitle{ABS-NF}
	\framesubtitle{Einführung}
	\setbeamertemplate{enumerate items}[default]
	Vorgehen:
	\begin{itemize}
		\item Any picewise linear scalar function has a so called min-max repräsentation
		\item All min max expressions can be expressed in terms of abs functions
		\item Picewise linearization wird erreicht durch algorithmic differentiation
	\end{itemize}
\end{frame}
%------------------------------------------------------------------------
\begin{frame}
	\frametitle{ABS-NF}
	\framesubtitle{Beispiel}
	\setbeamertemplate{enumerate items}[default]
	\begin{flalign*}
		F(x_1,x_2) &= (x_2^2 - x_1^+)^+ \\
			 (i)^+ &= \max(0, i)
	\end{flalign*}
	\begin{center}
		BILD
	\end{center}
	
\end{frame}
%------------------------------------------------------------------------
\begin{frame}
	\frametitle{ABS-NF}
	\framesubtitle{Einführung}
	\setbeamertemplate{enumerate items}[default]
	\begin{center}
		ADOL C, Beispiel
	\end{center}
\end{frame}
%------------------------------------------------------------------------
\begin{frame}
	\frametitle{ABS-NF}
	\framesubtitle{Aufgabenbeschreibung}
	\setbeamertemplate{enumerate items}[default]
	\begin{itemize}
		\item Evaluate abs normal form
		\item Calculate Gradient
		\item Solve abs-normal form system
	\end{itemize}
\end{frame}
%------------------------------------------------------------------------
\end{document}