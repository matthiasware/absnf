\section{Introcution}
\subsection{ABSNF}
The abs-normal form is a representation of piecewise linear (PL) functions. Any PL function can be transformed to obtain  this form. The process is described in []. The advantage and applications of the abs normal is not subject of this document but can be found in [] and [].

\begin{mydef*}
	ABS-Normal-Form \\
	For a PL function $f:\mathbb{R}^n \rightarrow \mathbb{R}^m$ the abs-normal representation takes the following form:
	\begin{flalign} \label{absnf}
	\begin{pmatrix}
	\Delta z \\
	\Delta y
	\end{pmatrix}
	= 
	\begin{pmatrix}
	a \\
	b
	\end{pmatrix}
	+
	\begin{pmatrix}
	Z & L \\
	J & Y 
	\end{pmatrix}
	\times
	\begin{pmatrix}
	\Delta x \\
	|\Delta z |
	\end{pmatrix}
	\end{flalign}
	where:
	\begin{flalign*}
		n,m,s \in \mathbb{R}, \Delta x \in \mathbb{R}^n, \Delta z \in \mathbb{R}^s, \Delta y \in \mathbb{R}^m, a \in \mathbb{R}^s, b \in \mathbb{R}^{m}, Z \in \mathbb{R}^{s\times n}, L \in \mathbb{R}^{s \times s}, Y \in \mathbb{R}^{m \times s}, J \in \mathbb{R}^{m \times n}
	\end{flalign*}
	and 
	\begin{flalign*}
		|\Delta z|
	\end{flalign*}
	is the element-wise absolute vector of $\Delta z$. $L$ is a lower triangular matrix.
\end{mydef*}
Note that this representation is almost a linear system of equations. The non-linear parts of the function are all captured in $|\Delta z|$.

\subsection{Problems}
Given a PL function in abs-normal-form. The following tasks were given:
\begin{enumerate}
	\item Evaluate the function in abs-normal form
	\begin{itemize}
		\item Geg: $a,b,Z,L,J,Y,\Delta x$
		\item Ges: $\Delta z, \Delta y$
	\end{itemize}
	\item Calculate the gradient of a function in abs-normal form
	\begin{itemize}
		\item Geg: $a,b,Z,L,J,Y, \Delta z$
		\item Ges: Gradient $\gamma, \Gamma$
	\end{itemize}
	\item Solve the system of equations
	\begin{itemize}
		\item Geg: $a,b,Z,L,J,Y,\Delta y$
		\item Ges: $\Delta x, \Delta Z$
	\end{itemize}
\end{enumerate}
Task was to solve each problem by using parallel computing with CUDA C++ to boost performance.
The specific questions were
\begin{enumerate}
	\item How can the problems be implemented with CUDA C++?
	\item Is there a benifit of using parallel computing?
	\item How does the implementation compare to a serial implementation?
	\item Performance?
\end{enumerate}
Each problem is content of one of the following sections, where we show the implementation and answer the given questions. [chapet 2] [chapter 3] [chapter 4]. \\

Additionally chapter [...] we describe our approach of choosing the right parameters for CUDA as well as show our kernnels.

Last but not least we discuss our solution and give some final thoughts and possible improvements and further questions.

\subsection{Additional Information}
\subsubsection{Used Software Libraries}
All the plots, prototypes and a serial implementation was done in Python 3.6, where numerics were done using numpy library. For the CUDA C++ implementation, we used the following libraries:
\begin{itemize}
	\item cuBLAS (cuda Basic Linear Algebra Subprograms)
	\begin{itemize}
		\item Matrix Vector operations
		\item Matrix Matrix operations
	\end{itemize}
	\item cuSOLVER
	\begin{itemize}
		\item Matrix factorization
		\item Triangular solve
	\end{itemize}
	\item C++ STL
\end{itemize}
\subsubsection{Devices}
We tested our code on the following devices, which were also used for benchmarking. The results can be found in chapter ...
\begin{itemize}
	\item NVIDIA Tesla P100
	\begin{itemize}
		\item Global Memory
		\item MPU
		\item Warps / MPU
		\item Threads / WARP
		\item MaxThreads / Block
		\item DOUBLE PRECISSION
		\item SINGLE PRECISSION
	\end{itemize}
	\item NVIDIA Geforce GTX 780
		\begin{itemize}
			\item Global Memory
			\item MPU
			\item Warps / MPU
			\item Threads / WARP
			\item MaxThreads / Block
			\item MaxFlops DOUBLE PRECISSION
			\item SINGLE PRECISSION
		\end{itemize}
	\item Intel Core i5-2500 CPU, 16GB RAM
\end{itemize}
\begin{center}
	TABLE
\end{center}
\subsubsection{Notation and Symbols}
In the rest of the document we use the following symbols and notation:
\begin{itemize}
	\item $m,n,s$ denote the dimensions of the datastructures of the abs-normal form (\label{absnf})
	\item $\Delta x, \Delta z, \Delta y, a,b ,Z ,L,J,Y$ denote the datastructures with given dimensions in ( \label{absnf}).
	\item $|\circ|$ is the absolute value of $\circ$ if $\circ$ is a scalar and the elementwise absolute vector if $\circ$ is a vector.
\end{itemize}

\begin{itemize}
	\item Tesla
	\item GTX
	\item i-5
\end{itemize}
\subsubsection{Project structure}
The code of the implementation as well as its corresponding unit-tests can be found ...
Plots, Serial implementation, Performance tests, raw data of performance ....