\documentclass{article}

%-----------------------------------------------------------------------------
%	Packages
%-----------------------------------------------------------------------------

%-----------------------------------------------------------------
\usepackage{animate}
\usepackage[utf8]{inputenc}	% für Umlaute ect.
\usepackage{fancyhdr} % für header
\usepackage{lastpage} % für footer
\usepackage{extramarks} % für header und footer
\usepackage{amsthm} % math stuff
\usepackage{amsmath} % math stuff
\usepackage{amssymb} % math stuff
\usepackage{color}
\usepackage{listings} % code listings
\usepackage{graphicx} % für graphics
\usepackage{color}
\usepackage{tikz}
\usepackage[absolute,overlay]{textpos} %to translate graphics through space
\usepackage{soul}
\usepackage{hyperref}
\usepackage{xcolor}
\usepackage{textpos}
\usepackage{caption}


\newcommand{\highlightred}[1]{%
	\colorbox{red!50}{$\displaystyle#1$}}
\newcommand{\highlightgreen}[1]{%
	\colorbox{green!50}{$\displaystyle#1$}}
\newcommand{\highlightblue}[1]{%
	\colorbox{blue!50}{$\displaystyle#1$}}
\newcommand{\highlightyellow}[1]{%
	\colorbox{yellow!50}{$\displaystyle#1$}}

%-----------------------------------------------------------------------------
%	Allegmeine Dokumentsettings
%-----------------------------------------------------------------------------
% Margins
\topmargin=-0.45in
\evensidemargin=0in
\oddsidemargin=0in
\textwidth=6.5in
\textheight=9.0in
\headsep=0.25in
%Header und Footer
\pagestyle{fancy}
\chead{\moduleTitle}  
\rhead{}
\lfoot{\lastxmark}
\cfoot{} 
\rfoot{\ \thepage\ of\ \protect\pageref{LastPage}}
\renewcommand\headrulewidth{0.4pt} % Size of the header rule
\renewcommand\footrulewidth{0.4pt} % Size of the footer rule

% Zeilenabstand
\linespread{1.0} 

%----------------------------------------------------------------------------------------
%	Commands and Environments
%----------------------------------------------------------------------------------------

%---- Allows vertical lines in matrices
\makeatletter
\renewcommand*\env@matrix[1][*\c@MaxMatrixCols c]{%
	\hskip -\arraycolsep
	\let\@ifnextchar\new@ifnextchar
	\array{#1}}
\makeatother
%--------------------------

\newcommand{\moduleTitle}{Projektarbeit - Graphische Modelle}
\newcommand{\moduleSubTitle}{Theoretische Informatík I}
\newcommand{\moduleClassInstructor}{Prof. Joachim Giesen}
\newcommand{\university}{FSU Jena}
\newcommand{\moduleSemester}{SS 17}
\newcommand{\authorName}{Matthias Mitterreiter}

% Für Code listings
\lstset{ %
	language=Python,                % choose the language of the code
	basicstyle=\footnotesize,       % the size of the fonts that are used for the code
	numbers=left,                   % where to put the line-numbers
	numberstyle=\footnotesize,      % the size of the fonts that are used for the line-numbers
	stepnumber=1,                   % the step between two line-numbers. If it is 1 each line will be numbered
	numbersep=5pt,                  % how far the line-numbers are from the code
	backgroundcolor=\color{white},  % choose the background color. You must add \usepackage{color}
	showspaces=false,               % show spaces adding particular underscores
	showstringspaces=false,         % underline spaces within strings
	showtabs=false,                 % show tabs within strings adding particular underscores
	frame=single,           % adds a frame around the code
	tabsize=2,          % sets default tabsize to 2 spaces
	captionpos=b,           % sets the caption-position to bottom
	breaklines=true,        % sets automatic line breaking
	breakatwhitespace=false,    % sets if automatic breaks should only happen at whitespacee
	escapeinside={\%*}{*)}          % if you want to add a comment within your code
}

% An environment for stpes, cases ect. 
% From: http://tex.stackexchange.com/questions/32798/a-step-by-step-environment
	\newenvironment{steps}[1]{\begin{enumerate}[label=#1 \arabic*]}{\end{enumerate}}
\makeatletter%
\def\step{%
	\@ifnextchar[ \@step{\@noitemargtrue\@step[\@itemlabel]}}
\def\@step[#1]{\item[#1]\mbox{}\\\hspace*{\dimexpr-\labelwidth-\labelsep}}
\makeatother

%http://tex.stackexchange.com/questions/10669/how-to-enumerate-equations
\def\Item$#1${\item $\displaystyle#1$
   \hfill\refstepcounter{equation}(\theequation)}

%-----------------------------------------------------------------------------
%	Theoreme
%-----------------------------------------------------------------------------
\theoremstyle{definition}
\newtheorem{mydef}{Definition}
\newtheorem*{mydef*}{Definition}
\newtheorem{mybei}{Beispiel}
\newtheorem*{mybei*}{Beispiel}
%------------------------------------------------------------------------
\newtheorem{mysatz}{Satz}
\newtheorem*{mysatz*}{Satz}
\newtheorem{mybew}{Beweis}

\newtheorem*{mybew*}{Beweis}
\newtheorem{myfolg}{Folgerung}
\newtheorem*{myfolg*}{Folgerung}
\newtheorem{mybemerk}{Bemerkung}
\newtheorem*{mybemerk*}{Bemerkung}

% Boxed Theoreme

\newenvironment{fmylemma*}
  {\begin{mdframed}\begin{mylemma*}}
  {\end{mylemma*}\end{mdframed}}

\newenvironment{fmykorollar*}
  {\begin{mdframed}\begin{mykorollar*}}
  {\end{mykorollar*}\end{mdframed}}

\newenvironment{fmysatz*}
  {\begin{mdframed}\begin{mysatz*}}
  {\end{mysatz*}\end{mdframed}}

\newenvironment{fmylemma}
  {\begin{mdframed}\begin{mylemma}}
  {\end{mylemma}\end{mdframed}}

\newenvironment{fmykorollar}
  {\begin{mdframed}\begin{mykorollar}}
  {\end{mykorollar}\end{mdframed}}

\newenvironment{fmysatz}
  {\begin{mdframed}\begin{mysatz}}
  {\end{mysatz}\end{mdframed}}


\definecolor{dkgreen}{rgb}{0,0.6,0}
\definecolor{dred}{rgb}{0.545,0,0}
\definecolor{dblue}{rgb}{0,0,0.545}
\definecolor{lgrey}{rgb}{0.9,0.9,0.9}
\definecolor{gray}{rgb}{0.4,0.4,0.4}
\definecolor{darkblue}{rgb}{0.0,0.0,0.6}
\lstdefinelanguage{cpp}{
	backgroundcolor=\color{lgrey},  
	basicstyle=\footnotesize \ttfamily \color{black} \bfseries,   
	breakatwhitespace=false,       
	breaklines=true,               
	captionpos=b,                   
	commentstyle=\color{dkgreen},   
	deletekeywords={...},          
	escapeinside={\%*}{*)},                  
	frame=single,                  
	language=C++,                
	keywordstyle=\color{purple},  
	morekeywords={BRIEFDescriptorConfig,string,TiXmlNode,DetectorDescriptorConfigContainer,istringstream,cerr,exit},
	ndkeywords={cudaMemcpy, cudaMemcpyDeviceToDevice, cublasDgemv, cublasDgemm},
	ndkeywordstyle=\color{blue},
	identifierstyle=\color{black},
	stringstyle=\color{blue},      
	numbers=left,                 
	numbersep=5pt,                  
	numberstyle=\tiny\color{black}, 
	rulecolor=\color{black},        
	showspaces=false,               
	showstringspaces=false,        
	showtabs=false,                
	stepnumber=1,                   
	tabsize=2,                     
	title=\lstname,                 
}

\begin{document}
	\section{TODO}
\begin{itemize}
	\item Double and No float
\end{itemize}

\section{Grid and Blocksize}
\section{Solve}
\section{Final Thoughts}
\begin{itemize}
	\item Improvements
	\item view
	\item Multidevice support
	\item Grid and Blocksize
\end{itemize}

\section{Anhang}
\begin{itemize}
	\item projectstructure (unittests ect, python prototypes)
	\item cublas check
\end{itemize}
	\section{Introcution}
\subsection{ABSNF}
The abs-normal form is a representation of piecewise linear (PL) functions. Any PL function can be transformed to obtain  this form. The process is described in []. The advantage and applications of the abs normal is not subject of this document but can be found in [] and [].

\begin{mydef*}
	ABS-Normal-Form \\
	For a PL function $f:\mathbb{R}^n \rightarrow \mathbb{R}^m$ the abs-normal representation takes the following form:
	\begin{flalign} \label{absnf}
	\begin{pmatrix}
	\Delta z \\
	\Delta y
	\end{pmatrix}
	= 
	\begin{pmatrix}
	a \\
	b
	\end{pmatrix}
	+
	\begin{pmatrix}
	Z & L \\
	J & Y 
	\end{pmatrix}
	\times
	\begin{pmatrix}
	\Delta x \\
	|\Delta z |
	\end{pmatrix}
	\end{flalign}
	where:
	\begin{flalign*}
		n,m,s \in \mathbb{R}, \Delta x \in \mathbb{R}^n, \Delta z \in \mathbb{R}^s, \Delta y \in \mathbb{R}^m, a \in \mathbb{R}^s, b \in \mathbb{R}^{m}, Z \in \mathbb{R}^{s\times n}, L \in \mathbb{R}^{s \times s}, Y \in \mathbb{R}^{m \times s}, J \in \mathbb{R}^{m \times n}
	\end{flalign*}
	and 
	\begin{flalign*}
		|\Delta z|
	\end{flalign*}
	is the element-wise absolute vector of $\Delta z$. $L$ is a lower triangular matrix.
\end{mydef*}
Note that this representation is almost a linear system of equations. The non-linear parts of the function are all captured in $|\Delta z|$.

\subsection{Problems}
Given a PL function in abs-normal-form. The following tasks were given:
\begin{enumerate}
	\item Evaluate the function in abs-normal form
	\begin{itemize}
		\item Geg: $a,b,Z,L,J,Y,\Delta x$
		\item Ges: $\Delta z, \Delta y$
	\end{itemize}
	\item Calculate the gradient of a function in abs-normal form
	\begin{itemize}
		\item Geg: $a,b,Z,L,J,Y, \Delta z$
		\item Ges: Gradient $\gamma, \Gamma$
	\end{itemize}
	\item Solve the system of equations
	\begin{itemize}
		\item Geg: $a,b,Z,L,J,Y,\Delta y$
		\item Ges: $\Delta x, \Delta Z$
	\end{itemize}
\end{enumerate}
Task was to solve each problem by using parallel computing with CUDA C++ to boost performance.
The specific questions were
\begin{enumerate}
	\item How can the problems be implemented with CUDA C++?
	\item Is there a benifit of using parallel computing?
	\item How does the implementation compare to a serial implementation?
	\item Performance?
\end{enumerate}
Each problem is content of one of the following sections, where we show the implementation and answer the given questions. [chapet 2] [chapter 3] [chapter 4]. \\

Additionally chapter [...] we describe our approach of choosing the right parameters for CUDA as well as show our kernnels.

Last but not least we discuss our solution and give some final thoughts and possible improvements and further questions.

\subsection{Additional Information}
\subsubsection{Used Software Libraries}
All the plots, prototypes and a serial implementation was done in Python 3.6, where numerics were done using numpy library. For the CUDA C++ implementation, we used the following libraries:
\begin{itemize}
	\item cuBLAS (cuda Basic Linear Algebra Subprograms)
	\begin{itemize}
		\item Matrix Vector operations
		\item Matrix Matrix operations
	\end{itemize}
	\item cuSOLVER
	\begin{itemize}
		\item Matrix factorization
		\item Triangular solve
	\end{itemize}
	\item C++ STL
\end{itemize}
\subsubsection{Devices}
We tested our code on the following devices, which were also used for benchmarking. The results can be found in chapter ...
\begin{itemize}
	\item NVIDIA Tesla P100
	\begin{itemize}
		\item Global Memory
		\item MPU
		\item Warps / MPU
		\item Threads / WARP
		\item MaxThreads / Block
		\item DOUBLE PRECISSION
		\item SINGLE PRECISSION
	\end{itemize}
	\item NVIDIA Geforce GTX 780
		\begin{itemize}
			\item Global Memory
			\item MPU
			\item Warps / MPU
			\item Threads / WARP
			\item MaxThreads / Block
			\item MaxFlops DOUBLE PRECISSION
			\item SINGLE PRECISSION
		\end{itemize}
	\item Intel Core i5-2500 CPU, 16GB RAM
\end{itemize}
\begin{center}
	TABLE
\end{center}
\subsubsection{Notation and Symbols}
In the rest of the document we use the following symbols and notation:
\begin{itemize}
	\item $m,n,s$ denote the dimensions of the datastructures of the abs-normal form (\label{absnf})
	\item $\Delta x, \Delta z, \Delta y, a,b ,Z ,L,J,Y$ denote the datastructures with given dimensions in ( \label{absnf}).
	\item $|\circ|$ is the absolute value of $\circ$ if $\circ$ is a scalar and the elementwise absolute vector if $\circ$ is a vector.
\end{itemize}
\subsubsection{Project structure}
The code of the implementation as well as its corresponding unit-tests can be found ...
Plots, Serial implementation, Performance tests, raw data of performance ....
	\section{Evaluation of the ABSNF}
\subsection{Problem Specification}

ABS-Normal Form:
\begin{flalign*}
\begin{pmatrix}
\Delta z \\
\Delta y
\end{pmatrix}
= 
\begin{pmatrix}
a \\
b
\end{pmatrix}
+
\begin{pmatrix}
Z & L \\
J & Y 
\end{pmatrix}
\circ
\begin{pmatrix}
\Delta x \\
|\Delta z |
\end{pmatrix}
\end{flalign*}
Given is a PL function in abs-nf. The evaluation of this function means calculating the vectors $\Delta y$ and $\Delta z$:
\begin{flalign}
\Delta y &= b + (J \times \Delta x) + (Y \times |\Delta z|) \label{eval_1}\\
\Delta z &= a + (Z \times \Delta x) + (L \times |\Delta z|) \label{eval_2}
\end{flalign}
where the following structures are given:
\begin{flalign*}
a,b,Z,L,J,Y,m,n,s, \Delta x
\end{flalign*}
In (\ref{eval_2}) $\Delta z$ depends on the element-wise absolute function of its own and therefore it cannot be calculated with a simple matrix vector dot product. Since the matrix $L$ is lower triangular, the vector $\Delta z$ can be iteratively calculated, by taking the row-wise dot product of $L$ and $|\Delta z|$. \\

\begin{flalign*}
k = a + Z \times \Delta x
\end{flalign*}

\begin{flalign*}
\highlightblue{\Delta  z_1}  &= \underbrace{L_1 \times |\Delta z|}_{=0} + k_1 = k_1 \\
\highlightyellow{\Delta z_2} &= L_2 \times |\Delta z| + k_2 \\
	&= L_{2,1} \times \highlightblue{|\Delta z_1 |} + k_2 \\
\highlightgreen{\Delta z_3} &= L_3 \times |\Delta z| + k_3 \\
	&= L_{3,1} \times \highlightblue{|\Delta z_1 |} + L_{3,2} \times \highlightyellow{|\Delta z_2|} + k_3 \\
\highlightred{\Delta z_4} &= L_{4} \times |\Delta z| + k_4 \\
	&= L_{4,1} \times \highlightblue{|\Delta z_1|} + 
	L_{4,2} \times \highlightyellow{|\Delta z_2|} +
	L_{4,3} \times \highlightgreen{|\Delta z_3|} + k_4 \\
	....
\end{flalign*}

\subsection{Implementation}
Our implementation is highly focused on speed and  demands the device to hold all the required data structures in global memory simultaneously. \\

Given this premise, the calculation of (\ref{eval_1}) and (\ref{eval_2}) is a series of dot products and therefore is highly parallelize-able. For this we relied mainly on CUBLAS routines. The implementation is available on [...] with several interfaces in [...]..

\begin{lstlisting}[language=cpp]
template <typename T>
void eval(T *a, T *b, 
T *Z, T *L, 
T *J, T *Y,
T *dx,
int m, int n, int s,
T *dz, T *dy,
T *abs_dz)
{
// dz = a
cudaMemcpy(dz, a, ., cudaMemcpyDeviceToDevice));
// dz = Z * dx + dx
cublasDgemv(.,Z, ., dx, . dz, .)
// dz[i] = L[i]_i * |dz|_i
for(int i=0; i<s; i++)
{
cublasDgemv( . ,&L[i * s], . ,abs_dz, . , &dz[i],.);
abs <<<1,1>>>(&dz[i], &abs_dz[i], 1);
}
// dy = b
cudaMemcpy(dy, b, ., cudaMemcpyDeviceToDevice);
// dy = dy + J*dx
cublasDgemv(.,J, ., dx, ., dy, .));
// dy = dy + Y * |dz|
cublasDgemv(., Y, ., abs_dz, ., dy, .));
}
\end{lstlisting}

\subsubsection{Performance Experiments}
For measuring performance and the subsequent analysis, we simplified the process by equalizing the dimensions of the data-structures:
\begin{flalign*}
	m = n = s
\end{flalign*}

\subsubsection{Single Execution}

In this experiment, we executed the serial python implementation as well as the parallel cuda implementation for different dimensions of $s$ and measured the total runtime of the program. The results can be seen in figure \ref{eval_single_repetition}. Since the results were not as expected in favor of the CUDA implementation, we also dismantled the runtime on the parallel devices and measured the "data-upload-time" and the "execution-time" separately. Those results can be found in figure \ref{eval_memory}. Here we can clearly see, that the transfer time, which is the time that it takes to upload the required datastaructures onto the device is extremely significant and takes a disproportional high share of the total time.

\begin{figure}[ht]
	\centering
	\includegraphics[width=0.6\textwidth]{img/eval_single_repetition.png}
	\caption{Single execution of the evaluation function on different devices.}
	\label{eval_single_repetition}
\end{figure}

\begin{figure}[ht]
	\centering
	\includegraphics[width=0.6\textwidth]{img/eval_memory.png}
	\caption{Datatransfer and execution time of the parallel implementation}
	\label{eval_memory}
\end{figure}

\subsubsection{Multiple Executions}

Since the scenario of a single execution of the evaluate function is not quite realistic, we crafted a second experiment, where we uploaded the data to the devices and executed the code 1000 times. Here we only measured the pure execution time without dataupload, which should be marginal with a high enough number of executions. The results can be found in figure \ref{eval_1000}.

\begin{figure}[ht]
	\centering
	\includegraphics[width=0.6\textwidth]{img/eval_mult_repetition.png}
	\caption{Multiple executions of the evaluate function on different devices}
	\label{eval_1000}
\end{figure}

\subsubsection{Analysis}
The results of the experiment are heavy in favor of the serial implementation.
For small data structures, that completely fit into the global memory of the device, we couldn't get any performance gains through the cuda implementation on given devices. fig \ref{eval_single_repetition} and  fig. \ref{eval_1000}. 
If the data structures get big enoguh such that they don't fit into the global memory of the device, we can expect the performance to be even worse, since the data-transfer time, takes a huge part of the overall runtime. figure \ref{eval_memory}.

We therefore came to the conclusion, that the considerable effort of implementing a parallel version of the eval function is not worth the effort.
\begin{itemize}
	\item Memory Complexity $O(s^2)$
	\item Complexity $O(s^2)$
	\item mempry is bottle neck
	\item no performance gain expected
\end{itemize}

what can be implemented differently? What can be improved?

\subsubsection{Notes}
\begin{itemize}
	\item Double Precision on GTX is nuts
\end{itemize}


	\section{Gradient}
\subsection{Problem Specification}

ABS-Normal Form:
\begin{flalign*}
\begin{pmatrix}
\Delta z \\
\Delta y
\end{pmatrix}
= 
\begin{pmatrix}
a \\
b
\end{pmatrix}
+
\begin{pmatrix}
Z & L \\
J & Y 
\end{pmatrix}
\times
\begin{pmatrix}
\Delta x \\
|\Delta z |
\end{pmatrix}
\end{flalign*}

\subsection{Implementation}
\subsection{Performance}
\subsection{Analysis}
\subsection{Notes}
	\section{Operations}

\begin{tabular}{c | c}
	Operation & Function \\
	\hline
	Matrix - Vector Product & cublas \\
	Matrix - Vector Vector Product & cublas \\
	Vector Vector Addition & cuutils
\end{tabular}
	%\begin{frame}
\frametitle{Inhalt}
\framesubtitle{...}
	\begin{enumerate}
		\item ABS - Normal Form
		\item Aufgabengeschreibung
		\item Implementierung
		\begin{enumerate}
			\item Verwendete Bibliotheken
			\item Evaluate
			\item Gradient
			\item Solve
		\end{enumerate}
		\item Anwendung
		\item Numerische Tests und Vergleiche
		\item Cool Snippets
		\item Imprevements
		\item Problems Lesson Learned
	\end{enumerate}
\end{frame}
%------------------------------------------------------------------------
\begin{frame}
\frametitle{ABS-NF}
\framesubtitle{Client Side Storage}
	\setbeamertemplate{enumerate items}[default]
	\begin{enumerate}
		\item Was ist das
		\item Wie bekommen ich das?
		\item Wofür brauche ich das?
	\end{enumerate}	
\end{frame}
%------------------------------------------------------------------------
\begin{frame}
\frametitle{Aufgabenbeschreibung}
\framesubtitle{Aufgabenbeschreibung}
\setbeamertemplate{enumerate items}[default]

\end{frame}
%------------------------------------------------------------------------
\begin{frame}
\frametitle{Implementierung}
\framesubtitle{Client Side Storage}
 So wenig wie möglich selbst machen \\
 CUBLAS, CUSOLVE, THRUST, C++ STL \\
 Genpgend Speicher vorhanden
\end{frame}
%------------------------------------------------------------------------
	%\section{Solve}
ABS-Normal Form:
\begin{flalign*}
	\begin{pmatrix}
		\Delta z \\
		\Delta y
	\end{pmatrix}
	= 
	\begin{pmatrix}
		a \\
		b
	\end{pmatrix}
	+
	\begin{pmatrix}
		Z & L \\
		J & Y 
	\end{pmatrix}
	\times
	\begin{pmatrix}
		\Delta x \\
		|\Delta z|
	\end{pmatrix}
\end{flalign*}

The last function, that had to be implemented was a solver for PL function in abs-normal form. Given:
\begin{flalign*}
	a,b,Z,L,J,Y,m,n,\Delta y
\end{flalign*}
We want to calculate:
\begin{flalign*}
	\Delta x, \Delta z
\end{flalign*}

\subsection{Deducing a solution}
First and foremost, we assume:
\begin{flalign*}
	\Delta y = 0
\end{flalign*}
It this is not the case, we can replace $b$ with $b'$:
\begin{flalign*}
	b' = b - \Delta y
\end{flalign*}
Now we can shift the equation system:
\begin{flalign*}
	\Delta y &= b + J \Delta x + Y |\Delta z| \\
	0 &= b + J \Delta x + Y |\Delta z| \\
	- b - Y |\Delta z| &= J \Delta x \\
	b + Y |\Delta z| &= J \Delta x (-1) \\
	J^{-1}(b + Y |\Delta z|) &= - \Delta x \\
	\Delta x = - J^{-1}(b + Y |\Delta z|)
\end{flalign*}

\begin{flalign*}
	\Delta z &= a + Z \Delta x + L |\Delta z| \\
	&= a + Z \Big( - J^{-1}(b + Y |\Delta z|) \Big) +  L |\Delta z| \\
	&= a + Z \Big( -J^{-1}b - J^{-1}Y|\Delta z| \Big) +  L |\Delta z| \\
	&= a - ZJ^{-1}b - Z J^{-1}Y|\Delta z| +  L |\Delta z| \\
	&= a - ZJ^{-1}b - (Z J^{-1}Y - L)|\Delta z|
\end{flalign*}
\begin{flalign*}
\Delta z &= a - ZJ^{-1}b - (Z J^{-1}Y - L)|\Delta z| \\
		 &= c + S|\Delta z| \\
c		 &= a - ZJ^{-1}b \\
S		 &= (L - Z J^{-1}Y)
\end{flalign*}

\subsection{Implementation}
\begin{flalign*}
	c &= a - ZJ^{-1}b \\
	S &= (L - Z J^{-1}Y)
\end{flalign*}
Need the inverse of $J$
\begin{itemize}
	\item Calculate cublas LU
	\item Cusolve QR
	\item LU -> Inverse
\end{itemize}
\begin{flalign*}
	c = a - Z* solve(J*X = b) \\
	S = L - Z* solve(J*X = Y) \\
\end{flalign*}

\end{document}