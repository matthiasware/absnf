\section{TODO}
\begin{itemize}
	\item Plot lineraization
	\item linearization
	\item linearization of abs in AD
\end{itemize}
\section{ABS-Normalform}
\begin{mydef}
	Smooth function \\
	A smooth function $f(x)$ is coninuous and has a continuous derivative
\end{mydef}
\begin{mybei}
	\begin{flalign*}
		f(x) = x^2, f'(x) =2x
	\end{flalign*}
	is smooth.
	\begin{flalign*}
		f(x) = |x| \\
		g'(x) = 1, x>0 \\
		f'(x) = 1, x<0 \\
	\end{flalign*}
	Therefore it is not smooth, since its derivative is not continuous.
\end{mybei}
\begin{mydef}
	A functin has a jump discontinuity at $x_0$ if 
	\begin{flalign*}
		\lim\limits_{x \rightarrow x_0^-} f(x) \not = \lim\limits_{x->x_0^+} f(x)
	\end{flalign*}
\end{mydef}
\begin{mybei}
	abs(x)
\end{mybei}
\begin{mydef}
	a picewise smooth function is made up of finetly many smooth functions separated by jump discontinuities.
\end{mydef}

Haben Picewise smooth function. Picewise linearization can be achieved in the style of algorithmic differentiation. by replacing all smooth elemental functions by their tangent line or plane.


\subsubsection{Assumptions}
\begin{itemize}
	\item All nonsmoothness can be cast in terms of the absolute value funtion
\end{itemize}

\subsubsection{Goal}
Gegben $F(X)$ wollen $F(X)=0$. Wobei $F(X)$ picewise linear in abs NF ist.

\subsubsection{Questions}
\begin{itemize}
	\item How do i get my representation?
	\item Finde minimum dieser funktionen?
	\item On Stable Picewise Linearization and Generalized Differentiation - Griewank
\end{itemize}
\subsection{Linearization}
Linearization is a linear approximation of a nonlinear system that is valid in a small region around an operating point.
\begin{flalign*}
	https://de.mathworks.com/help/slcontrol/ug/linearizing-nonlinear-models.html
\end{flalign*}
Linearization through tangent line at point $(a, f(a))$
\begin{flalign*}
	y &= f(x) \\
	m &= f'(a) \\
	y &= f(a) + f'(a)(x-a) \\
	y &= f'(a)x -f'(a)a + f(a) = f'(a)x + b \\
\end{flalign*}
\subsection{Anwendung}
Wenn Newton nicht haut mache. Zur Minimums findung von Funktion:
\begin{itemize}
	\item Stückweise Linearisierung in Punkt
	\item Berechne ABS Lösung
	\item Nutze Punkt für nächsten step
\end{itemize}
\subsection{Representation of Picewisefunctions with MAX-MIN}
Can represent every PL functin with min max functins
\subsection{Representation of MAX-MIN functions with abs functions}
\begin{flalign*}
	min(a,b) = \frac{1}{2} (a+b-abs(a-b)) \\
	max(a,b) = \frac{1}{2} (a+b+abs(a-b))
\end{flalign*}
\subsection{Absnormalform}
\subsection{Example}
Given is the following PL function:
\begin{flalign*}
	F(x_1,x_2) &= (x_2^2 - x_1^+)^+ \\
	(i)^+ &= \max(0, i)
\end{flalign*}

\subsubsection{ABS-Normalization}
First we rewrite the function such that picewise linearity is expressed with the aid of the abs function.
\begin{flalign*}
	F(x_1,x_2)
	&= (x_2^2 - x_1^+)^+ \\
	&= \max \big(0, x_2^2 - \max(0, x_1) \big) \\
	&= \frac{1}{2} \Big(0 + x_2^2 - \max(0, x_1) + abs(0 - x_2^2 + \max(0, x_1)) \Big) \\
	&= \frac{1}{2} \Big(0 + x_2^2 - \frac{1}{2}\Big[0 + x_1 + abs(0-x_1) \Big] + abs(0 - x_2^2 + \max(0, x_1)) \Big) \\
	&= \frac{1}{2} \Big(0 + x_2^2 - \frac{1}{2}\Big[0 + x_1 + abs(0-x_1) \Big] + abs\big(0 - x_2^2 + \frac{1}{2}(0+x_1+ abs(0-x_1))\big) \Big) \\
	&= \frac{1}{2} \Big(x_2^2 - \frac{1}{2}\Big[x_1 + abs(-x_1) \Big] + abs\big(-x_2^2 + \frac{1}{2}(x_1+ abs(-x_1))\big) \Big) \\
	&= \frac{1}{2} \Big(x_2^2 - \frac{1}{2}\Big[x_1 + abs(x_1) \Big] + abs\big(-x_2^2 + \frac{1}{2}(x_1+ abs(x_1))\big) \Big) \\
\end{flalign*}

\subsubsection{Straight-Line Code}
Next we bring the function in straight-line code representaion: \

\begin{flalign*}
	\begin{array}{rl|rl}
	\text{SLC} && & \text{Directional Derivatives} \\
	\hline
	w_1 &= x_1 	& \Delta w_1 &= \Delta x_1 \\
	w_2 &= x_2 		& \Delta w_2 &= \Delta x_2 \\
	\hline
	w_3 &= |w_1| 	& \Delta w_3 &= |\underbrace{w_1 + \Delta w_1}_{\Delta Z_1}| - |w_1| \\
	w_4 &= w_1 + w_3 & \Delta w_4 &= \Delta w_1 + \Delta w_3 \\
	w_5 &= \frac{1}{2} w_4 & \Delta w_5 &= \frac{1}{2} \Delta w_4 \\
	w_6 &= w_2 w_2 	& \Delta w_6 &= 2\Delta w_2 * w_2 \\
	w_7 &= w_5 - w_6 & \Delta w_7 &=  \Delta w_5 - \Delta w_6 \\
	w_8 &= |w_7| 	& \Delta w_8 &= |\underbrace{w_7 + \Delta w_7}_{\Delta Z_2}| - |w_7| \\
	w_9 &= w_6 - w_5 & \Delta w_9 &= \Delta w_6 - \Delta w_5 \\
	w_{10} &= w_9 + w_8 & \Delta w_{10} &= \Delta w_9 + \Delta w_9 \\
	w_{11} &= \frac{1}{2} w_{10} & \Delta w_{11} &= \frac{1}{2} \Delta w_{10} \\
	\hline
	y 	&= w_{11} 	& \Delta y &= \Delta w_{11} \\
	\end{array}
\end{flalign*}

Now we rewrite $Z_i$ and $Y_i$ such that they only depend on $x$. In the computitional graph, the equivalent operation is node contraction.
\begin{flalign*}
	\Delta Z_1 &= w_1 + \Delta w_1 = w_1 + \Delta x_1 \\
	\Delta Z_2 &= w_7 + \Delta w_7 \\
	&= w_7 + \Delta w_5 - \Delta w_6 \\
	&= w_7 + \frac{1}{2} \Delta w_4 - 2 \Delta w_2 w_2 \\
	&= w_7 + \frac{1}{2} (\Delta w_1 + \Delta w_3) - 2 \Delta x_2 w_2 \\
	&= w_7 + \frac{1}{2} (\Delta w_1 + |\Delta Z_1| - |w_1|) - 2 \Delta x_2 w_2 \\
	&= w_7 + \frac{1}{2} (\Delta x_1 + |\Delta Z_1| - |w_1|) - 2 \Delta x_2 w_2 \\
	&= w_7 + \frac{1}{2} \Delta x_1 + \frac{1}{2} |\Delta Z_1| - \frac{1}{2} |w_1| - 2 \Delta x_2 w_2 \\
	\Delta Y &= \Delta w_{11} \\
	&= \frac{1}{2}\Delta w_{10} \\
	&= \frac{1}{2}(\Delta w_9 + \Delta w_8) \\
	&= \frac{1}{2}(\Delta w_6 - \Delta w_5 + \Delta w_8) \\
	&= \frac{1}{2}(2\Delta w_2 w_2 - \frac{1}{2} \Delta w_4 + |\Delta Z_2| - |w_7|) \\
	&= \frac{1}{2}(2 \Delta x_2 w_2 - \frac{1}{2} (\Delta x_1 + |\Delta Z_1| - |w_1|)+ |\Delta Z_2| - |w_7| ) \\
	&= \frac{1}{2}(2 \Delta x_2 w_2 - \frac{1}{2} (\Delta x_1 + |\Delta Z_1| - |w_1|)+ |\Delta Z_2| - |w_7| ) \\
	&= \Delta x_2 w_2 - \frac{1}{4} \Delta x_1 - \frac{1}{4}|\Delta Z_1| + \frac{1}{4} |w_1| + \frac{1}{2}|\Delta Z_2| - \frac{1}{2} |w_7|
\end{flalign*}

\begin{flalign*}
	\begin{pmatrix}
		\Delta Z_1 \\
		\Delta Z_2 \\
		\hline
		\Delta Y
	\end{pmatrix}
	&= 
	\begin{pmatrix}
		w_1 \\
		w_7 - \frac{1}{2} |w_1| \\
		\hline
		\frac{1}{4} |w_1| - \frac{1}{2} |w_7|
	\end{pmatrix}
	+ 
	\begin{pmatrix}
		1 \Delta x_1 + 0 \Delta x_2 \\
		\frac{1}{2} \Delta x_1 + 0 \Delta x_2 \\
		\hline 
		- \frac{1}{4} \Delta x_1 + w_2 \Delta x_2 
	\end{pmatrix}
	+ 
	\begin{pmatrix}
	 0 |\Delta Z_1| + 0 |\Delta Z_2| \\
	 \frac{1}{2} |\Delta Z_1| + 0 \Delta Z_2 \\
	 \hline 
	 - \frac{1}{4} |\Delta Z_1| + \frac{1}{2} \Delta Z_2
	\end{pmatrix} \\
	&= 
	\begin{pmatrix}
	w_1 \\
	w_7 - \frac{1}{2} |w_1| \\
	\hline
	\frac{1}{4} |w_1| - \frac{1}{2} |w_7|
	\end{pmatrix} +
	\begin{pmatrix}[cc|cc]
		1 & 0  & 0 & 0 \\
		\frac{1}{2} & 0 & \frac{1}{2} & 0 \\
		\hline 
		- \frac{1}{4} & w_2 & - \frac{1}{4} & \frac{1}{2}
	\end{pmatrix}
	\times
	\begin{pmatrix}
		\Delta x_1 \\
		\Delta x_2 \\
		\hline
		|\Delta Z_1 | \\
		|\Delta Z_2 |
	\end{pmatrix}
\end{flalign*}

\subsubsection{ABS-Equation-System}

As a last step we bring the function in the form of a linear equation system.

\begin{flalign*}
	\begin{pmatrix}
		\Delta Z \\
		\Delta Y
	\end{pmatrix}
	= 
	\begin{pmatrix}
		a \\
		b
	\end{pmatrix}
	+
	\begin{pmatrix}
		Z & L \\
		J & Y 
	\end{pmatrix}
	\times
	\begin{pmatrix}
		\Delta x \\
		\Delta z
	\end{pmatrix}
\end{flalign*}

\section{Evaluation}
Input: $a, b, Z, L, J, Y, n, m, s, \Delta x$ \\
Output: $\Delta y, \Delta Z$
\section{Gradient}
\section{Solve}

\section{Calculations}
\subsection{Gradient of abs()}
\begin{flalign*}
	f(x) &= |x| \\
	\nabla f(x) &= |x + \nabla x| - |x| \\
	&= \max\{x + \nabla x, -x - \nabla x \} - \max\{x, -x\} \\
\end{flalign*}
\begin{itemize}
	\item $x>0$
	\begin{itemize}
		\item $ \nabla x > 0$
		\begin{flalign*}
			\nabla f(x) = x + \nabla x - x = \nabla x
		\end{flalign*}
		\item $ \nabla x < 0$
		\begin{itemize}
			\item $\nabla x > - x$
			\begin{flalign*}
				\nabla f(x) = x + \nabla x - x = \nabla x
			\end{flalign*}
			\item $\nabla x < - x$
			\begin{flalign*}
				\nabla f(x) &= \max\{x + \nabla x, -x - \nabla x \} - \max\{x, -x\} \\
				&= - x - \nabla x - x  = -2x - \nabla x
			\end{flalign*}
			\item $\nabla x = - x$
			\begin{flalign*}
				\nabla f(x) &= x + \nabla x - x = \nabla x \\
				&= - x - \nabla x - x = \nabla x - \nabla x + \nabla x = \nabla x \\
			\end{flalign*}
		\end{itemize}
		\item $ \nabla x = 0$
		\begin{flalign*}
			\nabla f(x) = x + \nabla x - x = \nabla x = 0
		\end{flalign*}
	\end{itemize}
	\item $x<0$
	\begin{itemize}
		\item $ \nabla x > 0$
		\item $ \nabla x < 0$
		\item $ \nabla x = 0$
	\end{itemize}	
	\item $x=0$
	\begin{flalign*}
		???
	\end{flalign*}
\end{itemize}


\section{Example ABS}
Let
\begin{flalign*}
	F(x) = 3 | x- 1 | = 3 abs(x-1)
\end{flalign*}
The task is to find $\Delta F(x, \Delta x)$, so that:
\begin{flalign*}
	F(x + \Delta x) \approx F(x) + \Delta F(x, \Delta x)
\end{flalign*}
therefore
\begin{flalign*}
	F(x) = |x| \Rightarrow \Delta F'(x, \Delta x) = |x+\Delta x| - |x|
\end{flalign*}
now 
\begin{flalign*}
	F(x) + \Delta F(x, \Delta x) 
	&= |x| + |x + \Delta x| - |x| \\
	&= |x + \Delta x| \\
	&= F(x + \Delta x)
\end{flalign*}

\begin{mybei}
	\begin{flalign*}
		F(x) &= 3|x-1|+4 \\
		\Delta F(x, \Delta x) &= 3|(x-1) + \Delta x| - 3|x-1| \\
		F(x) + \Delta F(x, \Delta x) &= 3|x-1| + 4 + 3|(x-1) + \Delta x| - 3|x-1| \\
		&=  3|(x-1) + \Delta x|  + 4 \\
		&= F(x + \Delta x)
	\end{flalign*}
\end{mybei}