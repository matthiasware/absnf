\usepackage{animate} % needed for animations and videos
\usepackage[utf8]{inputenc}	% für Umlaute ect.
\usepackage{fancyhdr} % für header
\usepackage{lastpage} % für footer
\usepackage{extramarks} % für header und footer
\usepackage{amsthm} % math stuff
\usepackage{amsmath} % math stuff
\usepackage{amssymb} % math stuff
\usepackage{color}
\usepackage{listings} % code listings
\usepackage{graphicx} % für graphics
\usepackage{color}
\usepackage{tikz}
\usepackage[absolute,overlay]{textpos} %to translate graphics through space
\usepackage{soul}
\usepackage{hyperref}
\usepackage{xcolor}
\usepackage{textpos}
\usepackage{caption}
\usepackage{parcolumns}
\usepackage{enumerate}
\usepackage[english]{babel}
\usepackage[section]{placeins} %forces placeins to stay in section
\usepackage{datetime} % cusom dates
\usepackage{afterpage}

%===========================================================
% Dates
%===========================================================

\newdateformat{monthyeardate}{%
	\monthname[\THEMONTH] \THEYEAR}

%===========================================================
% Matrixhighlightning
%===========================================================

\newcommand{\highlightred}[1]{%
	\colorbox{red!50}{$\displaystyle#1$}}
\newcommand{\highlightgreen}[1]{%
	\colorbox{green!50}{$\displaystyle#1$}}
\newcommand{\highlightblue}[1]{%
	\colorbox{blue!50}{$\displaystyle#1$}}
\newcommand{\highlightyellow}[1]{%
	\colorbox{yellow!50}{$\displaystyle#1$}}

%===========================================================
% Codelistings
%===========================================================

\definecolor{dkgreen}{rgb}{0,0.6,0}
\definecolor{dred}{rgb}{0.545,0,0}
\definecolor{dblue}{rgb}{0,0,0.545}
\definecolor{lgrey}{rgb}{0.9,0.9,0.9}
\definecolor{gray}{rgb}{0.4,0.4,0.4}
\definecolor{darkblue}{rgb}{0.0,0.0,0.6}
\lstdefinelanguage{cpp}{
	backgroundcolor=\color{lgrey},  
	basicstyle=\footnotesize \ttfamily \color{black} \bfseries,   
	breakatwhitespace=false,       
	breaklines=true,               
	captionpos=b,                   
	commentstyle=\color{dkgreen},   
	deletekeywords={...},          
	escapeinside={\%*}{*)},                  
	frame=single,                  
	language=C++,                
	keywordstyle=\color{purple},  
	morekeywords={BRIEFDescriptorConfig,string, blockDim, gridDim, threadIdx, blockIdx},
	ndkeywords={cudaMemcpy, cudaMemcpyDeviceToDevice, cublasDgemv,
			    cublasDgemm, abs, initTss, initIdentity, getTriangularInverse, multWithDz, doSomething},
	ndkeywordstyle=\color{blue},
	identifierstyle=\color{black},
	stringstyle=\color{blue},      
	numbers=left,                 
	numbersep=5pt,                  
	numberstyle=\tiny\color{black}, 
	rulecolor=\color{black},        
	showspaces=false,               
	showstringspaces=false,        
	showtabs=false,                
	stepnumber=1,                   
	tabsize=2,                     
	title=\lstname,                 
}

%===========================================================
% Package:
%===========================================================
% Allows vertical lines in matrices
\makeatletter
\renewcommand*\env@matrix[1][*\c@MaxMatrixCols c]{%
	\hskip -\arraycolsep
	\let\@ifnextchar\new@ifnextchar
	\array{#1}}
\makeatother


%===========================================================
% Package:
%===========================================================

% An environment for stpes, cases ect. 
% From: http://tex.stackexchange.com/questions/32798/a-step-by-step-environment
\newenvironment{steps}[1]{\begin{enumerate}[label=#1 \arabic*]}{\end{enumerate}}
\makeatletter%
\def\step{%
	\@ifnextchar[ \@step{\@noitemargtrue\@step[\@itemlabel]}}
\def\@step[#1]{\item[#1]\mbox{}\\\hspace*{\dimexpr-\labelwidth-\labelsep}}
\makeatother

%http://tex.stackexchange.com/questions/10669/how-to-enumerate-equations
\def\Item$#1${\item $\displaystyle#1$
	\hfill\refstepcounter{equation}(\theequation)}

%===========================================================
% Package:
%==========================================================
% Theoreme
\theoremstyle{definition}
\newtheorem{mydef}{Definition}
\newtheorem*{mydef*}{Definition}
\newtheorem{mybei}{Beispiel}
\newtheorem*{mybei*}{Beispiel}
%------------------------------------------------------------------------
\newtheorem{mysatz}{Satz}
\newtheorem*{mysatz*}{Satz}
\newtheorem{mybew}{Beweis}

\newtheorem*{mybew*}{Beweis}
\newtheorem{myfolg}{Folgerung}
\newtheorem*{myfolg*}{Folgerung}
\newtheorem{mybemerk}{Bemerkung}
\newtheorem*{mybemerk*}{Bemerkung}

% Boxed Theoreme

\newenvironment{fmylemma*}
{\begin{mdframed}\begin{mylemma*}}
		{\end{mylemma*}\end{mdframed}}

\newenvironment{fmykorollar*}
{\begin{mdframed}\begin{mykorollar*}}
		{\end{mykorollar*}\end{mdframed}}

\newenvironment{fmysatz*}
{\begin{mdframed}\begin{mysatz*}}
		{\end{mysatz*}\end{mdframed}}

\newenvironment{fmylemma}
{\begin{mdframed}\begin{mylemma}}
		{\end{mylemma}\end{mdframed}}

\newenvironment{fmykorollar}
{\begin{mdframed}\begin{mykorollar}}
		{\end{mykorollar}\end{mdframed}}

\newenvironment{fmysatz}
{\begin{mdframed}\begin{mysatz}}
		{\end{mysatz}\end{mdframed}}
