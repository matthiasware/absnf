\section{Introduction}
TODO
\begin{itemize}
	\item in jedem kapitel bezug auf introduction fragen nehmen
	\item Titlepage with graphics
	\item Plot of eval where mismeasured
\end{itemize}


This is an documentation and project review. blabla

\subsection{The ABS-Normal Form}
The abs-normal form is a representation of piecewise linear (PL) functions. Any PL function can be transformed and represented within this form. The process of transforming a PL function into abs-normal form is described in \cite{Griewank2013} and \cite{Griewank2017}. Applications and properties of the abs-normal form is not subject of this document but can be found in \cite{Griewank2017}.

\begin{mydef*}
	ABS-Normal Form \\
	For a PL function $f:\mathbb{R}^n \rightarrow \mathbb{R}^m$ the abs-normal representation takes the following form:
	\begin{flalign} \label{absnf}
	\begin{pmatrix}
	\Delta z \\
	\Delta y
	\end{pmatrix}
	= 
	\begin{pmatrix}
	a \\
	b
	\end{pmatrix}
	+
	\begin{pmatrix}
	Z & L \\
	J & Y 
	\end{pmatrix}
	\times
	\begin{pmatrix}
	\Delta x \\
	|\Delta z |
	\end{pmatrix}
	\end{flalign}
	where:
	\begin{flalign*}
		n,m,s \in \mathbb{R}, \Delta x \in \mathbb{R}^n, \Delta z \in \mathbb{R}^s, \Delta y \in \mathbb{R}^m, a \in \mathbb{R}^s, b \in \mathbb{R}^{m}, Z \in \mathbb{R}^{s\times n}, L \in \mathbb{R}^{s \times s}, Y \in \mathbb{R}^{m \times s}, J \in \mathbb{R}^{m \times n}
	\end{flalign*}
	and 
	\begin{flalign*}
		|\Delta z|
	\end{flalign*}
	is the element-wise absolute vector of $\Delta z$. $L$ is a lower triangular matrix.
\end{mydef*}
Note that this representation is almost a linear system of equations. The non-linear parts of the function are all captured in $|\Delta z|$.

\subsection{Problems}
Given a PL function in abs-normal form. The following problems were given:
\begin{enumerate}
	\item Evaluate a function in abs-normal form:
	\begin{itemize}
		\item Given: $a,b,Z,L,J,Y,\Delta x$
		\item Wanted: $\Delta z, \Delta y$
	\end{itemize}
	\item Calculate the gradient of a function in abs-normal form:
	\begin{itemize}
		\item Given: $a,b,Z,L,J,Y, \Delta z$
		\item Wanted: Gradient $\gamma, \Gamma$
	\end{itemize}
	\item Solve the system of equations of a function in abs-normal form:
	\begin{itemize}
		\item Given: $a,b,Z,L,J,Y,\Delta y$
		\item Wanted: $\Delta x, \Delta Z$
	\end{itemize}
\end{enumerate}

\subsection{Tasks}
The aim of this project was to solve each of these problems by using parallel computing with CUDA C++ to boost performance.
In particular, for each of the problems the following questions had to answered:
\begin{enumerate}[{(}I{)}]
	\item How can the problem be solved theoretically?
	\item How can the problems be implemented with CUDA C++?
	\item Is there a benefit of using parallel computing?
	\item How does the implementation perform on different devices?
\end{enumerate}

\subsection{Outline}
In the following sections we answer each question for all of the given problems.
Section \ref{sec_evaluation} deals with the evaluation of functions in abs-normal form, where we show that a parallel implementation might not be beneficial. In section \ref{sec_gradient} we address the problem of calculating the gradient. Those results were quite promising. In section \ref{sec_solve} we show one possibility of solving a function in abs-normal form.

Last but not least we discuss our solution and give some final thoughts and possible improvements and further questions.

In the appendix we listed all devices, that we used to benchmark our implementation \ref{sec_devides}. Used software libraries are described in \ref{sec_libraries} and our notations as well abbrevations and symbols are documented in \ref{sec_notation}.
