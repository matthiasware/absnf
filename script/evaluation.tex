\section{Evaluation of the ABSNF}
\begin{itemize}
	\item Problem
	\item what is parallelizable?
	\item Implementations
	\item Performance
	\begin{itemize}
		\item Single Core Float and Double
		\item GTX Float and Double
		\item Tesla Float and Double
	\end{itemize}
	\item Review and Notes
\end{itemize}

\subsection{Problem Specification}

ABS-Normal Form:
\begin{flalign*}
\begin{pmatrix}
\Delta z \\
\Delta y
\end{pmatrix}
= 
\begin{pmatrix}
a \\
b
\end{pmatrix}
+
\begin{pmatrix}
Z & L \\
J & Y 
\end{pmatrix}
\times
\begin{pmatrix}
\Delta x \\
|\Delta z |
\end{pmatrix}
\end{flalign*}
Given is a PL function in abs-nf. The evaluation of this function means calculating the vectors $\Delta y$ and $\Delta z$:
\begin{flalign}
\Delta y &= b + (J \times \Delta x) + (Y \times |\Delta z|) \label{eval_1}\\
\Delta z &= a + (Z \times \Delta x) + (L \times |\Delta z|) \label{eval_2}
\end{flalign}
where the following structures are given:
\begin{flalign*}
a,b,Z,L,J,Y,m,n,s, \Delta x
\end{flalign*}
In (\ref{eval_2}) $\Delta z$ depends on the element-wise absolute function of its own and therefore it cannot be calculated straightforward. Since the matrix $L$ is lower triangular, the vector $\Delta z$ can be iteratively calculated, by taking the row-wise dotproduct of $L$ and $|\Delta z|$. \\

\begin{flalign*}
k = a + Z \times \Delta x
\end{flalign*}

\begin{flalign*}
\highlightblue{\Delta  z_1}  &= \underbrace{L_1 \times |\Delta z|}_{=0} + k_1 = k_1 \\
\highlightyellow{\Delta z_2} &= L_2 \times |\Delta z| + k_2 \\
	&= L_{2,1} \times \highlightblue{|\Delta z_1 |} + k_2 \\
\highlightgreen{\Delta z_3} &= L_3 \times |\Delta z| + k_3 \\
	&= L_{3,1} \times \highlightblue{|\Delta z_1 |} + L_{3,2} \times \highlightyellow{|\Delta z_2|} + k_3 \\
\highlightred{\Delta z_4} &= L_{4} \times |\Delta z| + k_4 \\
	&= L_{4,1} \times \highlightblue{|\Delta z_1|} + 
	L_{4,2} \times \highlightyellow{|\Delta z_2|} +
	L_{4,3} \times \highlightgreen{|\Delta z_3|} + k_4 \\
	....
\end{flalign*}

\subsection{Implementation}
Our implementation is highly focused on speed and  demands the device to hold all the required data structures in global memory simultaneously. \\

Given this premise, the calculation of (\ref{eval_1}) and (\ref{eval_2}) is a series of dot products and therefore is highly parallelize-able. For this we relied mainly on CUBLAS routines. The implementation is available on [...] with several interfaces in [...].

\subsubsection{Performance}
For measuring performance and the subsequent analysis, we simplified the process by equalizing the dimensions of the datasstuctures:
\begin{flalign*}
	m = n = s
\end{flalign*}

\subsubsection{Single Execution}

\subsubsection{Multiple Executions}

\begin{itemize}
	\item one iteration with memory
	\item multiple executions
\end{itemize}

\subsubsection{Analysis}
We are interested in performance 
For simplicity reasons we assume $n=m=s$ and obtain the following results:
\begin{itemize}
	\item Memory Complexity $O(s^2)$
	\item Complexity $O(s^2)$
\end{itemize}

\begin{itemize}
	\item Double Precision on GTX is nuts
\end{itemize}

